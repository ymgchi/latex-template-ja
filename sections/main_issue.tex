%! TEX root = ../main.tex
\documentclass[main]{subfiles}

\begin{document}

\section{目的}
この実験・演習の目的を簡潔に記述する。

\section{方法}
\subsection{使用機器・環境}
使用した機器や実験環境について記述する。

例:
\begin{itemize}
    \item コンピュータ:◯◯社製、型番ABC-123
    \item ソフトウェア:△△ Ver.1.0
    \item その他:必要に応じて追加
\end{itemize}

\subsection{実験・演習の手順}
どのような手順で実験や演習を行ったかを具体的に記述する。

\section{結果}
実験や演習で得られた結果を記述する。図表を活用すると分かりやすい。

% 図の例(図は下部にタイトル)
% \begin{figure}[htbp]
%     \centering
%     \includegraphics[keepaspectratio, width=0.8\linewidth]{figures/sample.pdf}
%     \caption{図のタイトルと簡単な説明}
%     \label{fig:sample}
% \end{figure}

% 表の例(表は上部にタイトル)
% \begin{table}[htbp]
%     \centering
%     \caption{表のタイトルと簡単な説明}
%     \label{tab:sample}
%     \begin{tabular}{|c|c|c|}
%         \hline
%         項目1 & 項目2 & 項目3 \\
%         \hline
%         データ1 & データ2 & データ3 \\
%         \hline
%     \end{tabular}
% \end{table}

% 数式の例
% \begin{equation}
%     E = mc^2
%     \label{eq:sample}
% \end{equation}

% 本文中で図表を参照する例:
% Fig.~\ref{fig:sample}に示すように、...
% Table~\ref{tab:sample}より、...
% Eq.~\ref{eq:sample}から、...

\section{考察}
結果に対する考察を記述する。以下のような観点から考察するとよい。

\begin{itemize}
    \item 予想した結果と実際の結果の比較
    \item 誤差や予想と異なる結果が生じた原因
    \item 他の方法との比較
    \item 結果の応用可能性
    \item 関連する理論についての考察
\end{itemize}

注意:単なる感想や実験内容の説明、言い訳は考察ではない。

\section{まとめ}
実験・演習全体を通して分かったことを簡潔にまとめる。

% 参考文献を使用する場合はコメント解除
% \section{参考文献}
% \begin{enumerate}
%     \item 著者名「書籍名」、出版社、発行年
%     \item 著者名「論文タイトル」、学会誌名、巻号、ページ、発行年
%     \item URL: https://example.com (アクセス日: 2025年11月25日)
% \end{enumerate}

\end{document}
